\section{Output Elements}
Output elements take care of inter-process communication.
These elements can use specific communication means,
depending on the type of simulation they are used for,
and can communicate specific types of data.

\subsection{Stream output}\label{sec:EL:OUTELEM:STREAM_OUTPUT}
This is a special element which takes care of sending output
to external processes by means of either \kw{local} or \kw{inet} sockets
during batch or real-time simulations.
This topic is under development, so expect frequent changes, and
please do not count too much on backward compatibility.

The syntax is:
%\begin{verbatim}
\begin{Verbatim}[commandchars=\\\{\}]
    \bnt{elem_type} ::= \kw{stream output}

    \bnt{arglist} ::=
        \kw{stream name} , " \bnt{stream_name} " ,
            [ \kw{send after} , \{ \kw{convergence} | \kw{predict} \} , ]
        \kw{create} , \{ \kw{yes} | \kw{no} \} ,
        [ \{ \kw{local} , " \bnt{socket_name} " , |
            [ \kw{port} , \bnt{port_number} , ]
            [ \kw{host} , " \bnt{host_name} " , ] \} ]
        [ \kw{socket type} , \{ \kw{tcp} | \kw{udp} \} , ]
        [ \{ [ \kw{no} ] \kw{signal}
            | [ \kw{non} ] \kw{blocking}
            | [ \kw{no} ] \kw{send first}
            | [ \kw{do not} ] \kw{abort if broken} \} [ , ... ] , ]
        [ \kw{output every} , \bnt{steps} , ]
        [ \kw{echo} , \bnt{file_name}
            [ , \kw{precision} , \bnt{precision} ]
            [ , \kw{shift} , \bnt{shift} ] , ]
        \bnt{content}
\end{Verbatim}
%\end{verbatim}
The stream output allows MBDyn to send streamed outputs 
to remote processes during both batch and real-time simulations,
using sockets either in the \texttt{local} or in the \texttt{inet} namespace.
If the simulation is run in real-time using RTAI, RTAI mailboxes
are used instead.

\begin{itemize}
\item \nt{stream\_name} is the name of the RTAI mailbox where 
the output is written  (a unique string no more than six characters long);
it is basically ignored by the \kw{stream output} element
except when using RTAI;

\item \kw{send after} specifies when the output is sent.
By default, it is sent after \kw{convergence}; as an alternative, after \kw{predict} can be used
to send the predicted values instead.

\item the \kw{create} keyword determines whether the socket
must be created or looked-up as already existing on the system;
if \kw{create} is set to \kw{no}, MBDyn will retry for 60 seconds
and then give up;

\item \nt{socket\_name} is the path of the \kw{local} socket 
that is used to communicate between processes;

\item \nt{port\_number} is the port number to be used with a \texttt{inet} socket.
The default port number is 9011 (intentionally unassigned by IANA).
If no \nt{host\_name} is given, \kw{localhost} will be used;

\item \nt{host\_name} is the name or the IP of the remote host where
the mailbox resides; note that if this field is given, \kw{create} must
be set to \kw{no}.
The simulation will not start until the socket is created on the remote host;

\item \kw{socket type} defaults to \kw{tcp};

\item the flag \kw{no signal} requests that no \texttt{SIGPIPE} be raised
when sending through a socket when the other end is broken
(by default, \texttt{SIGPIPE} is raised);

\item the flag \kw{non blocking} requests that operations on the socket
do not block (or block, in case of \kw{blocking}, the default);

\item the flag \kw{no send first} requests that no send occurs before
the first time step (by default, data are always sent);

\item the flag \kw{do not abort if broken} requests that the simulation
continues in case the connection breaks.
No further data send will occur for the duration of the simulation
(the default);

\item the field \kw{output every} requests output to occur
only every \nt{steps};

\item the field \kw{echo} causes the content that is sent to the peer
to be echoed on file \nt{file\_name}; the optional parameter \nt{precision}
determines the number of digits used in the output; the optional parameter
\nt{shift} is currently unused;

\item the field \nt{content} is detailed in the next section.
\end{itemize}
This element, when used with the \kw{motion} content type,
obsoletes the \kw{stream motion output} element
(see Section~\ref{sec:EL:OUTELEM:STREAM_MOTION_OUTPUT}).
When the simulation is executed in real-time using RTAI,
this element obsoletes the \kw{RTAI output} element
(see Section~\ref{sec:EL:OUTELEM:RTAI_out}).



\subsubsection{Streamed output}
\label{sec:EL:OUTELEM:STREAM_OUTPUT:STREAMED_OUTPUT}
Different types of data can be sent.
The most general form is called \kw{values}, consisting
in an arbitrary set of independent scalar channels.
A form specifically intended to communicate the motion
of a mechanical system is called \kw{motion}, consisting
in a subset of the kinematics of a set of structural nodes:
%\begin{verbatim}
\begin{Verbatim}[commandchars=\\\{\}]
    \bnt{content} ::= \{ \bnt{values} | \bnt{motion} | \bnt{user_defined} \}

    \bnt{values} ::= [ \kw{values} , ]
        \bnt{channel_number} ,
            \bnt{value}
            [ , ... ]
        [ , \kw{modifier} , \bnt{content_modifier} ]

    \bnt{value} ::=
        \{ [ \kw{nodedof} , ] (\hty{NodeDof}) \bnt{output_dof}
            | \kw{drive} , (\hty{DriveCaller}) \bnt{drive_data} \}

    \bnt{motion} ::= \kw{motion} ,
        [ \kw{output flags} ,
            \{ \kw{position}
                | \kw{orientation matrix}
                | \kw{orientation matrix transpose}
                | \kw{velocity}
                | \kw{angular velocity} \}
             [ , ... ] , ]
        \{ \kw{all} | \bnt{struct_node_label} [ , ... ] \}

    \bnt{user_defined} ::= \bnt{user_defined_type} [ , ... ]
\end{Verbatim}
%\end{verbatim}
where
\begin{itemize}
\item the (optional) keyword \kw{values} indicates that a set
of independent scalar channels is output by the element;

\item the number of channels \nt{channel\_number} that are written
determines how many \nt{value} entries must be read.
In case of \kw{nodedof} (the default, deprecated),
they must be valid scalar dof entries, which can be connected
in many ways to nodal degrees of freedom;
in case of \kw{drive}, they can be arbitrary functions,
including node or element private data;

\item the keyword \kw{modifier} defines how the content of the stream
is modified before being sent;
see Section~\ref{sec:Stream} for details.

\item the keyword \kw{motion} indicates that a subset of the kinematic
parameters of a set of structural nodes is output by the element.
As opposed to the \nt{values} case, which is intended for generic
interprocess communication output, this content type is intended to ease
and optimize the output of the motion of structural nodes,
to be used for on-line visualization purposes.
By default, only the position of the selected nodes is sent.
This is intended for interoperability with a development version
of EasyAnim which can read the motion info (the so-called ``van'' file)
from a stream.
The optional keyword \kw{output flags} allows to request the output
of specific node kinematics: the node position, orientation matrix
(either row- or column-major), the velocity and the angular velocity.
The default is equivalent to \texttt{\kw{output flags}, \kw{position}}.
\end{itemize}



\subsubsection{Non real-time simulation}
During non real-time simulations, streams operate in blocking mode.
The meaning of the parameters is:
\begin{itemize}
\item \nt{stream\_name} indicates the name the stream would be known as
by RTAI; it must be no more than 6 characters long, and mostly useless;

\item the instruction \kw{create} determines whether MBDyn will create
the socket, or try to connect to an existing one;

\item the keyword \kw{local} indicates that a socket 
in the local namespace will be used; if \kw{create} is set to \kw{yes},
the socket is created, otherwise it must exist.

\item either of the keywords \kw{port} or \kw{host} indicate that a socket
in the internet namespace will be used;

if \kw{create} is set to \kw{yes}, \nt{host\_name} indicates 
the host that is allowed to connect to the socket; it defaults 
to any host (\texttt{0.0.0.0}); if \kw{create} is set to \kw{no},
\nt{host\_name} indicates what host to connect to; the default 
is localhost (\texttt{127.0.0.1}); the default port is \kw{9011}
(intentionally unassigned by IANA);

\item the flag \kw{no signal} is honored;

\item the flag \kw{non blocking} is honored;

\item the flag \kw{no send first} is honored;

\item the flag \kw{do not abort if broken} is honored.
\end{itemize}
If no socket type is specified, i.e.\ none of the \kw{local}, \kw{port} 
and \kw{host} keywords are given, a socket is opened by default 
in the internet namespace with the default IP and port; the \kw{create}
keyword is mandatory.



\subsubsection{Real-time simulation}
During real-time simulations, streams wrap non-blocking RTAI mailboxes.
The meaning of the parameters is:
\begin{itemize}
\item the parameter \nt{stream\_name} indicates the name the stream
will be known as in RTAI's resource namespace; it must be exactly 6 characters long;

\item the instruction \kw{create} determines whether the mailbox will be
created or looked for by MBDyn;

\item the keyword \kw{local} is ignored;

\item the keyword \kw{host} indicates that a mailbox on a remote host 
will be used; it is useless when \kw{create} is set to \kw{yes}, because
RTAI does not provide the possibility to create remote resources;
if none is given, a local mailbox is assumed;

\item the keyword \kw{port} is ignored.

\item the flag \kw{no signal} is ignored;

\item the flag \kw{non blocking} is honored; however, blocking mailboxes
make little sense, and real-time synchronization using RTAI should not rely
on blocking mailboxes;

\item the flag \kw{no send first} is ignored (although it should be honored
when the mailbox is blocking);

\item the flag \kw{do not abort if broken} is ignored;
the program is always terminated if a mailbox is no longer available.
\end{itemize}





\subsection{RTAI output}\label{sec:EL:OUTELEM:RTAI_out}
This element is actually used only when the simulation is scheduled
using RTAI; otherwise, the corresponding \kw{stream output} element
is used (see Section~\ref{sec:EL:OUTELEM:STREAM_OUTPUT}).
As a consequence, its explicit use is discouraged and deprecated.
The \kw{stream output} element should be used instead.




\subsection{Stream motion output}\label{sec:EL:OUTELEM:STREAM_MOTION_OUTPUT}
This element type is obsoleted by the \kw{stream output} element
with the \kw{motion} content type
(see Section~\ref{sec:EL:OUTELEM:STREAM_MOTION_OUTPUT}).
The syntax is:
%\begin{verbatim}
\begin{Verbatim}[commandchars=\\\{\}]
    \bnt{elem_type} ::= \kw{stream motion output}

    \bnt{arglist} ::= 
        \kw{stream name} , " \bnt{stream_name} " ,
        \kw{create} , \{ \kw{yes} | \kw{no} \} ,
        [ \{ \kw{local} , " \bnt{socket_name} " ,
                | [ \kw{port} , \bnt{port_number} , ] [ \kw{host} , " \bnt{host_name} " , ] \} ]
        [ \{ [ \kw{no} ] \kw{signal}
                | [ \kw{non} ] \kw{blocking}
                | [ \kw{no} ] \kw{send first}
                | [ \kw{do not} ] \kw{abort if broken} \}
            [ , ... ] , ]
        \bnt{motion}
\end{Verbatim}
%\end{verbatim}
Its support may be discontinued in the future.


%\subsection{Structural output}
%\label{sec:EL:OUTELEM:STRUCTURAL_OUTPUT}
%TODO.


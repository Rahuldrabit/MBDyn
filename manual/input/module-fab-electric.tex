% MBDyn (C) is a multibody analysis code. 
% http://www.mbdyn.org
% 
% Copyright (C) 1996-2023
% 
% Pierangelo Masarati	<pierangelo.masarati@polimi.it>
% Paolo Mantegazza	<paolo.mantegazza@polimi.it>
% 
% Dipartimento di Ingegneria Aerospaziale - Politecnico di Milano
% via La Masa, 34 - 20156 Milano, Italy
% http://www.aero.polimi.it
% 
% Changing this copyright notice is forbidden.
% 
% This program is free software; you can redistribute it and/or modify
% it under the terms of the GNU General Public License as published by
% the Free Software Foundation (version 2 of the License).
% 
% 
% This program is distributed in the hope that it will be useful,
% but WITHOUT ANY WARRANTY; without even the implied warranty of
% MERCHANTABILITY or FITNESS FOR A PARTICULAR PURPOSE.  See the
% GNU General Public License for more details.
% 
% You should have received a copy of the GNU General Public License
% along with this program; if not, write to the Free Software
% Foundation, Inc., 59 Temple Place, Suite 330, Boston, MA  02111-1307  USA
% 
% Author: Eduardo Okabe


\subsection{Module-fab-electric}
\label{sec:MODULE:FAB-ELECTRIC}
\emph{Author: Eduardo Okabe}

\noindent
This module implements several electric components.


\subsubsection{BJT}
\begin{Verbatim}[commandchars=\\\{\}]
    \bnt{name} ::= \kw{bjt}

    \bnt{module_data} ::=
        \{ \kw{npn} | \kw{pnp} \} ,
        \bnt{collector_electric_node_label} ,
        \bnt{base_electric_node_label} ,
        \bnt{emitter_electric_node_label} ,
        (\hty{real}) \bnt{base_to_emitter_leakage_saturation_current} ,
        (\hty{real}) \bnt{base_to_collector_leakage_saturation_current} ,
        (\hty{real}) \bnt{ideal_maximum_forward_beta} ,
        (\hty{real}) \bnt{ideal_maximum_reverse_beta} ,
        [ , \kw{thermal voltage} , (\hty{real}) \bnt{thermal_voltage} ]
\end{Verbatim}


\subsubsection{Capacitor}
\begin{Verbatim}[commandchars=\\\{\}]
    \bnt{name} ::= \kw{capacitor}

    \bnt{module_data} ::=
        \bnt{electric_node_1_label} ,
        \bnt{electric_node_2_label} ,
        (\hty{real}) \bnt{capacitance}
\end{Verbatim}

\paragraph{Output.}
\label{sec:MODULE:FAB-ELECTRIC:CAPACITOR:OUTPUT}
The format is:
\begin{itemize}
\item the label of the element;
\item the current in the element (\kw{I} in NetCDF format);
%\item the voltage on the first node (\kw{V1} in NetCDF format);
%\item the voltage on the second node (\kw{V2} in NetCDF format).
\end{itemize}


\subsubsection{Diode}
\begin{Verbatim}[commandchars=\\\{\}]
    \bnt{name} ::= \kw{diode}

    \bnt{module_data} ::=
        \bnt{electric_node_1_label} ,
        \bnt{electric_node_2_label} ,
        (\hty{real}) \bnt{forward_saturation_current} ,
        (\hty{real}) \bnt{forward_ideality_factor} ,
        (\hty{real}) \bnt{breakdown_current} ,
        (\hty{real}) \bnt{breakdown_voltage} ,
        (\hty{real}) \bnt{reverse_ideality_factor}
        [ , \kw{thermal voltage} , (\hty{real}) \bnt{thermal_voltage} ]
\end{Verbatim}


\subsubsection{Electrical source}
\begin{Verbatim}[commandchars=\\\{\}]
    \bnt{name} ::= \kw{electrical source}

    \bnt{module_data} ::=
        \{ \kw{current} | \kw{voltage} \} ,
            [ \kw{control} , \{ \kw{current} | \kw{voltage} \} ,
                \bnt{input_electric_node_1_label} ,
                \bnt{input_electric_node_2_label} , ]
        \bnt{output_electric_node_1_label} ,
        \bnt{output_electric_node_2_label} ,
        (\hty{DriveCaller}) \bnt{source_driver}
\end{Verbatim}


\subsubsection{Ideal transformer}
\begin{Verbatim}[commandchars=\\\{\}]
    \bnt{name} ::= \kw{ideal transformer}

    \bnt{module_data} ::=
        \bnt{input_electric_node_1_label} ,
        \bnt{input_electric_node_2_label} ,
        \bnt{output_electric_node_1_label} ,
        \bnt{output_electric_node_2_label} ,
        (\hty{DriveCaller}) \bnt{tranformer_ratio}
\end{Verbatim}


\subsubsection{Inductor}
\begin{Verbatim}[commandchars=\\\{\}]
    \bnt{name} ::= \kw{inductor}

    \bnt{module_data} ::=
        \bnt{electric_node_1_label} ,
        \bnt{electric_node_2_label} ,
        (\hty{real}) \bnt{inductance}
\end{Verbatim}

\paragraph{Output.}
\label{sec:MODULE:FAB-ELECTRIC:INDUCTOR:OUTPUT}
The format is:
\begin{itemize}
\item the label of the element;
\item the current in the element (\kw{I} in NetCDF format);
%\item the voltage on the first node (\kw{V1} in NetCDF format);
%\item the voltage on the second node (\kw{V2} in NetCDF format).
\end{itemize}


\subsubsection{Operational amplifier}
\begin{Verbatim}[commandchars=\\\{\}]
    \bnt{name} ::= \kw{operational amplifier}

    \bnt{module_data} ::=
        \bnt{input_electric_node_1_label} ,
        \bnt{input_electric_node_2_label} ,
        \bnt{reference_electric_node_1_label} ,
        \bnt{output_electric_node_2_label}
        [ , \kw{gain} , (\hty{real}) \bnt{gain} ]
        [ , \kw{input resistance} , (\hty{real}) \bnt{input_resistance} ]
\end{Verbatim}


\subsubsection{Proximity sensor}
\begin{Verbatim}[commandchars=\\\{\}]
    \bnt{name} ::= \kw{proximity sensor}

    \bnt{module_data} ::=
        \bnt{struct_node_1_label} ,
            [ , \kw{position}, (\hty{Vec3}) \bnt{node_1_offset} , ]
        \bnt{struct_node_2_label} ,
            [ , \kw{position}, (\hty{Vec3}) \bnt{node_2_offset} , ]
        \bnt{electric_node_1_label} ,
        \bnt{electric_node_2_label} ,
        (\hty{ScalarFunction}) \bnt{distance_to_voltage_conversion_function}
\end{Verbatim}


\subsubsection{Resistor}
\label{sec:MODULE:FAB-ELECTRIC:RESISTOR}
\begin{Verbatim}[commandchars=\\\{\}]
    \bnt{name} ::= \kw{resistor}

    \bnt{module_data} ::=
        \bnt{electric_node_1_label} ,
        \bnt{electric_node_2_label} ,
        (\hty{real}) \bnt{resistance}
\end{Verbatim}

\paragraph{Output.}
\label{sec:MODULE:FAB-ELECTRIC:RESISTOR:OUTPUT}
The format is:
\begin{itemize}
\item the label of the element;
\item the current in the element (\kw{I} in NetCDF format);
%\item the voltage on the first node (\kw{V1} in NetCDF format);
%\item the voltage on the second node (\kw{V2} in NetCDF format).
\end{itemize}


\subsubsection{Switch}
\begin{Verbatim}[commandchars=\\\{\}]
    \bnt{name} ::= \kw{switch}

    \bnt{module_data} ::=
        \bnt{electric_node_1_label} ,
        \bnt{electric_node_2_label} ,
        (\hty{DriveCaller}) \bnt{switch_state}
\end{Verbatim}
where \nt{switch\_state} is a drive that determines the state of the switch (0: open, 1: closed).

\noindent
\emph{NOTE: untested!}




% $Header$
% Copyright (C) 1996-2023 Pierangelo Masarati <pierangelo.masarati@polimi.it>
% Dipartimento di Ingegneria Aerospaziale, Politecnico di Milano
%
% Parentesi: tonde, quadre, curly, dritte, doppie e angolari.
\newcommand{\plbr}[1]{ \left( #1 \right) }
\newcommand{\sqbr}[1]{ \left[ #1 \right] }
\newcommand{\cubr}[1]{ \left\{ #1 \right\} }
\newcommand{\shbr}[1]{ \left| #1 \right| }
\newcommand{\nrbr}[1]{ \left\| #1 \right\| }
\newcommand{\anbr}[1]{ \langle #1 \rangle }

% Parentesi solo a sinistra: tonde, quadre, curly, dritte, doppie e angolari.
\newcommand{\lplbr}[1]{ \left( #1 \right. }
\newcommand{\lsqbr}[1]{ \left[ #1 \right. }
\newcommand{\lcubr}[1]{ \left\{ #1 \right. }
\newcommand{\lshbr}[1]{ \left| #1 \right. }
\newcommand{\lnrbr}[1]{ \left\| #1 \right. }
\newcommand{\lanbr}[1]{ \langle #1 \right. }

% Parentesi solo a destra: tonde, quadre, curly, dritte,doppie e angolari.
\newcommand{\rplbr}[1]{ \left. #1 \right) }
\newcommand{\rsqbr}[1]{ \left. #1 \right] }
\newcommand{\rcubr}[1]{ \left. #1 \right\} }
\newcommand{\rshbr}[1]{ \left. #1 \right| }
\newcommand{\rnrbr}[1]{ \left. #1 \right\| }
\newcommand{\ranbr}[1]{ \left. #1 \rangle }

% Vettori verticali:
\newcommand{\vvect}[2]{ \begin{array}{ #1 } #2 \end{array} }
\newcommand{\cvvect}[1]{ \begin{array}{c} #1 \end{array} }
\newcommand{\lvvect}[1]{ \begin{array}{l} #1 \end{array} }
\newcommand{\rvvect}[1]{ \begin{array}{r} #1 \end{array} }

% Vettori orizzontali:
\newcommand{\hvect}[2]{ \begin{array}{ #1 } #2 \end{array} }

% Matrici:
\newcommand{\matr}[2]{ \begin{array}{ #1 } #2 \end{array} }

% Integrali: uso \intg{inf}{sup}{arg}{dvar}
\newcommand{\intg}[4]{ \int_{#1}^{#2} {#3} \ {#4} }

% Limite: uso \limt{var}{lim}{arg}
\newcommand{\limt}[3]{ \lim_{{#1} \rightarrow {#2}} {#3}}

% LogLike functions
\newcommand{\llk}[1]{\ensuremath{\mathrm{#1}}}

\newcommand{\diag}[0]{\llk{diag}}
\newcommand{\tr}[0]{\llk{tr}}
\newcommand{\sym}[0]{\llk{sym}}
\newcommand{\skw}[0]{\llk{skw}}

\newcommand{\step}[0]{\llk{step}}
\newcommand{\imp}[0]{\llk{imp}}

\newcommand{\grad}[0]{\llk{grad}}
\newcommand{\divr}[0]{\llk{div}}
\newcommand{\rot}[0]{\llk{rot}}

% In italiano ...
\newcommand{\sca}[0]{\llk{sca}}

% first, second, etc
\newcommand{\first}[0]{1\ensuremath{^{\mathrm{st}}}}    % 1^st
\newcommand{\second}[0]{2\ensuremath{^{\mathrm{nd}}}}   % 2^nd
\newcommand{\third}[0]{3\ensuremath{^{\mathrm{rd}}}}    % 3^rd
\newcommand{\rth}[0]{\ensuremath{^{\mathrm{th}}}}       %  ^th

\newcommand{\degr}[0]{\ensuremath{^{\mathrm{o}}}}

% esponenziale
\providecommand{\e}[1]{\llk{e}^{#1}}

\documentclass{article}

\usepackage{graphicx}
\usepackage{url}

\oddsidemargin 0.cm \textwidth 16.5cm \topmargin 0.cm \textheight 22cm

\begin{document}

\title{The Princeton beam experiment: \\ analysis using MBDyn}
\author{Pierangelo Masarati}
\date{}
\maketitle

\section{Introduction}
This document presents the analysis of the so-called
Princeton beam experiment \cite{DOWELL-1975-PRINCETON-1194,DOWELL-1975-PRINCETON-1257}
using MBDyn \url{<http://www.mbdyn.org/>}.
Refer to \url{<http://www.dymoresolutions.com/Benchmarks/PrincetonBeam/PrincetonBeamDescription.pdf>}
for more details.

\section{Results}
Figure~\ref{fig:tip} presents the tip displacement and rotation
for three load conditions.
The results have been obtained using MBDyn 1.5.4.
The model consists of 5 three-node beam elements.
The input deck is available here: \url{<http://www.mbdyn.org/documentation/examples/princeton.tar.gz>}.

\begin{figure}
\centering
\includegraphics[width=.33\textwidth]{princeton_u2}\hfill%
\includegraphics[width=.33\textwidth]{princeton_u3}\hfill%
\includegraphics[width=.33\textwidth]{princeton_phi}
\caption{Flapwise displacement (left),
lagwise displacement (center),
and twist (right)
at the beam tip versus loading angle
for three loading conditions.}
\label{fig:tip}
\end{figure}


\bibliographystyle{unsrt}
\bibliography{../../../manual/mybib}

\end{document}


\documentclass{article}
\usepackage{graphicx}
\usepackage{url}
\usepackage{amsmath}
\usepackage{amssymb}
\usepackage{psfrag}
\usepackage{subfigure}
\usepackage[a4paper, text={14.5cm,22.0cm},centering]{geometry}

\newcommand{\mesh}[2]{#1$\times$#2}

\begin{document}

\title{Application of MBDyn's Shell Element \\ to Standard Benchmark Problems}
\author{Pierangelo Masarati \\ \url{masarati@aero.polimi.it}}
\date{\today}
\maketitle

\section{Introduction}
MBDyn 1.3 provides a four-node shell element
implemented by Marco Morandini and Riccardo Vescovini.
This document analyzes some of the benchmarks
discussed in \cite{SZE-2004}:

\begin{itemize}
\item Test 3.1: Cantilever subjected to end shear force
\item Test 3.2: Cantilever subjected to end moment
\item Test 3.3: Slit annular plate subjected to lifting line force
\item Test 3.4: Hemispherical shell subjected to alternating radial forces (todo)
\item Test 3.5: Pullout of an open-ended cylindrical shell
\item Test 3.6: Pinched cylindrical shell mounted over rigid diaphragms
\item Test 3.7: Pinched semi-cylindrical isotropic and laminated shells (todo)
\item Test 3.8: Hinged cylindrical isotropic and laminated roofs (todo)
\end{itemize}



\subsection{Test 3.1: Cantilever subjected to end shear force}
\paragraph{Files:}
\begin{itemize}
\item \verb!sze31_8x1!: model with \mesh{8}{1} mesh
\item \verb!sze31_16x1!: model with \mesh{16}{1} mesh
\item \verb!sze31_ref.dat!: results from \cite{SZE-2004} (\mesh{16}{1} mesh)
\item \verb!sze31_8x1.dat!: results with \mesh{8}{1} mesh
\item \verb!sze31_16x1.dat!: results with \mesh{16}{1} mesh
\item \verb!sze31.plt!: plot script
\item \verb!sze31.sh!: plot preparation script
\end{itemize}

\paragraph{Data:}
\begin{itemize}
\item $E$ = 1.2$\times$10$^6$
\item $\nu$ = 0
\item $L$ = 10
\item $b$ = 1
\item $h$ = 0.1
\item $P_0$ = $EI/L^2$ = 1
\item $P_\text{max}$ = $4P_0$ = 4
\end{itemize}

\begin{figure}[h]
\centering
\psfrag{-Utip}{\footnotesize $-U_{\text{tip}}$}
\psfrag{Wtip}{\footnotesize $W_{\text{tip}}$}
\psfrag{16x1}{\footnotesize \mesh{16}{1}}
\includegraphics[width=.7\textwidth]{sze31/sze31}
\caption{Test 3.1: Cantilever subjected to end shear force}
\label{fig:sze31}
\end{figure}

\paragraph{To produce plot:}
\begin{itemize}
\item run \verb!mbdyn -f sze31_16x1!
\item run \verb!sze31.sh sze31_16x1!
\item run \verb!gnuplot sze31.plt!
\end{itemize}

\paragraph{Notes:}
\begin{itemize}
\item The $8\times1$ and $16\times1$ meshes produce nearly identical results.
\end{itemize}

\paragraph{Dynamic test:}
First frequency of cantilever beam:
\begin{equation*}
	\omega_0
	=
	\left(\frac{1.8751}{L}\right)^2
	\sqrt{\frac{EI}{\rho A}}
\end{equation*}
Density for desired frequency:
\begin{equation*}
	\rho
	=
	\left(\frac{1.8751}{L}\right)^4
	\frac{EI}{\omega_0^2 A}
\end{equation*}
\begin{itemize}
\item $A$ = $bh$ = 0.1
\item $EI$ = $P L^2$ = 100 (from \cite{SZE-2004})
\item $\omega_0$ = 10 Hz = 62.8 rad/s
\item $\rho$ = 3.1314$\times$10$^{-4}$
\end{itemize}
File \verb!sze31_8x1_dyn! (\mesh{8}{1} mesh):
\begin{itemize}
\item direct eigenanalysis: 10.0355 Hz
\item time integration of free vibrations: 10.03148 Hz ($\Delta t$ = 0.001 s).
\end{itemize}



\subsection{Test 3.2: Cantilever subjected to end moment}

\paragraph{Files:}
\begin{itemize}
\item \verb!sze32_8x1!: model with \mesh{8}{1} mesh
\item \verb!sze32_16x1!: model with \mesh{16}{1} mesh
\item \verb!sze32_ref.dat!: results from \cite{SZE-2004} (\mesh{16}{1} mesh \& exact)
\item \verb!sze32_8x1.dat!: results with \mesh{8}{1} mesh
\item \verb!sze32_16x1.dat!: results with \mesh{16}{1} mesh
\item \verb!sze32.plt!: plot script
\item \verb!sze32.sh!: plot preparation script
\end{itemize}

\paragraph{Data:}
\begin{itemize}
\item $E$ = 1.2$\times$10$^6$
\item $\nu$ = 0
\item $L$ = 12
\item $b$ = 1
\item $h$ = 0.1
\item $M_0$ = $EI/L^2$ = 25/3
\item $M_\text{max}$ = $4M_0$ = 50$\pi$/3
\end{itemize}

\begin{figure}[h]
\centering
\psfrag{-Utip}{\footnotesize $-U_{\text{tip}}$}
\psfrag{Wtip}{\footnotesize $W_{\text{tip}}$}
\psfrag{16x1}{\footnotesize \mesh{16}{1}}
\includegraphics[width=.7\textwidth]{sze32/sze32}
\caption{Test 3.2: Cantilever subjected to end moment}
\label{fig:sze32}
\end{figure}

\paragraph{To produce plot:}
\begin{itemize}
\item run \verb!mbdyn -f sze32_16x1!
\item run \verb!sze32.sh sze32_16x1!
\item run \verb!gnuplot sze32.plt!
\end{itemize}

\paragraph{Notes:}
\begin{itemize}
\item shell4eas does not converge past 20\% of Mmax
\item shell4easans does not converge past 60\% of Mmax
\item results with \mesh{8}{1} and \mesh{16}{1} do not differ significantly
\end{itemize}


\subsection{Test 3.3: Slit annular plate subjected to lifting line force}


\paragraph{Files:}
\begin{itemize}
\item \verb!sze33!: generic model
\item \verb!sze33.awk!: generic model generation script
\item \verb!sze33_ref.dat!: results from \cite{SZE-2004} (\mesh{10}{80} mesh)
\item \verb!sze33_6x30.dat!: results with \mesh{6}{30} mesh
\item \verb!sze33_10x80.dat!: results with \mesh{10}{80} mesh
\item \verb!sze33.plt!: plot script
\item \verb!sze33.sh!: plot preparation script
\end{itemize}

\paragraph{Data:}
\begin{itemize}
\item $E$ = 21$\times$10$^6$
\item $\nu$ = 0
\item $R_i$ = 6
\item $R_o$ = 10
\item $h$ = 0.03
\item $M_0$ = $EI/L^2$ = 25/3
\item $P_\text{max}$ = 0.8 (force/length)
\end{itemize}

\begin{figure}[h]
\centering
\psfrag{WA}{\footnotesize $W_\text{A}$}
\psfrag{WB}{\footnotesize $W_\text{B}$}
\psfrag{6x30}{\footnotesize \mesh{6}{30}}
\psfrag{10x80}{\footnotesize \mesh{10}{80}}
\includegraphics[width=.7\textwidth]{sze33/sze33}
\caption{Test 3.3: Slit annular plate subjected to lifting line force}
\label{fig:sze33}
\end{figure}

\paragraph{To produce plot:}
\begin{itemize}
\item for \verb!mesh! in \verb!"6x30" "10x80"! ; do
	\begin{itemize}
	\item run \verb!awk -f sze33.awk -v mesh=$mesh!
	\item run \verb!mbdyn -f sze33 -o sze33_$mesh!
	\item run \verb!sze33.sh sze33_$mesh!
	\end{itemize}
\item run \verb!gnuplot sze33.plt!
\end{itemize}

\paragraph{Notes:}
\begin{itemize}
\item mesh \mesh{10}{80} quite close to reference
\item mesh \mesh{6}{30} differs appreciably
\end{itemize}


\subsection{Test 3.4: Hemispherical shell subjected to alternating radial forces}

\paragraph{Notes:}
\begin{itemize}
\item not implemented; see \verb!tests/plates/HemiShell*!
\end{itemize}


\subsection{Test 3.5: Pullout of an open-ended cylindrical shell}

\paragraph{Files:}
\begin{itemize}
\item \verb!sze35!: generic model
\item \verb!sze35.awk!: generic model generation script
\item \verb!sze35_ref.dat!: results from \cite{SZE-2004} (\mesh{24}{36} mesh)
\item \verb!sze35_16x24.dat!: results with \mesh{16}{24} mesh
\item \verb!sze35_24x36.dat!: results with \mesh{24}{36} mesh
\item \verb!sze35.plt!: plot script
\item \verb!sze35.sh!: plot preparation script
\end{itemize}

\paragraph{Data:}
\begin{itemize}
\item $E$ = 10.5$\times$10$^6$
\item $\nu$ = 0.3125
\item $R$ = 4.953
\item $L$ = 10.35
\item $h$ = 0.094
\item $P_\text{max}$ = 40,000
\end{itemize}

\begin{figure}[h]
\centering
\psfrag{WA}{\footnotesize $W_\text{A}$}
\psfrag{-UB}{\footnotesize $-U_\text{B}$}
\psfrag{-UC}{\footnotesize $-U_\text{C}$}
\psfrag{16x24}{\footnotesize \mesh{16}{24}}
\psfrag{24x36}{\footnotesize \mesh{24}{36}}
\subfigure[Imposed force at point A]{%
	\includegraphics[width=.49\textwidth]{sze35/sze35}%
	\label{fig:sze35}%
}\hfill%
\subfigure[Imposed displacement at point A]{%
	\includegraphics[width=.49\textwidth]{sze35/sze35_d}%
	\label{fig:sze35_d}%
}
\caption{Test 3.5: Pullout of an open-ended cylindrical shell}
\end{figure}

\paragraph{To produce plot:}
\begin{itemize}
\item for \verb!mesh! in \verb!"16x24" "24x36"! ; do
	\begin{itemize}
	\item run \verb!awk -f sze35.awk -v mesh=$mesh!
	\item run \verb!mbdyn -f sze35 -o sze35_$mesh!
	\item run \verb!sze35.sh sze35_$mesh!
	\end{itemize}
\item run \verb!gnuplot sze35.plt!
\end{itemize}

\paragraph{Notes:}
\begin{itemize}
\item results with mesh \mesh{16}{24} and \mesh{24}{36} quite close to each other
\item results differ significantly from reference
\item snapthrough occurs at load 25\% lower
\item with the EAS ANS implementation the simulation no longer converges after the snapthrough
\item with the EAS implementation, the analysis does not converge right before the snapthrough.
\end{itemize}
When the displacement of point A is imposed:
\begin{itemize}
\item the simulations run smoothly to the end (Figure~\ref{fig:sze35_d})
\item the snapthrough results in an appreciable droop of required force
\item the results with the \mesh{16}{24} and the \mesh{24}{36} mesh
are essentially identical up to the snapthrough
\end{itemize}



\subsection{Test 3.6: Pinched cylindrical shell mounted over rigid diaphragms}

\paragraph{Files:}
\begin{itemize}
\item \verb!sze36!: generic model
\item \verb!sze36.awk!: generic model generation script
\item \verb!sze36_ref.dat!: results from \cite{SZE-2004} (\mesh{48}{48} mesh)
\item \verb!sze36_40x40.dat!: results with \mesh{40}{40} mesh
\item \verb!sze36_48x48.dat!: results with \mesh{48}{48} mesh
\item \verb!sze36.plt!: plot script
\item \verb!sze36.sh!: plot preparation script
\end{itemize}

\paragraph{Data:}
\begin{itemize}
\item $E$ = 30$\times$10$^3$
\item $\nu$ = 0.3
\item $R$ = 100
\item $L$ = 200
\item $h$ = 1
\item $P_\text{max}$ = 12,000
\end{itemize}

\begin{figure}[h]
\centering
\psfrag{-WA}{\footnotesize $-W_\text{A}$}
\psfrag{UB}{\footnotesize $U_\text{B}$}
\psfrag{40x40}{\footnotesize \mesh{40}{40}}
\psfrag{48x48}{\footnotesize \mesh{48}{48}}
\subfigure[Imposed force at point A]{%
	\includegraphics[width=.49\textwidth]{sze36/sze36}%
	\label{fig:sze36}%
}\hfill%
\subfigure[Imposed displacement at point A]{%
	\includegraphics[width=.49\textwidth]{sze36/sze36_d}%
	\label{fig:sze36_d}%
}
\caption{Test 3.6: Pinched cylindrical shell mounted over rigid diaphragms}
\end{figure}

\paragraph{To produce plot:}
\begin{itemize}
\item for \verb!mesh! in \verb!"40x40" "48x48"! ; do
	\begin{itemize}
	\item run \verb!awk -f sze36.awk -v mesh=$mesh!
	\item run \verb!mbdyn -f sze36 -o sze36_$mesh!
	\item run \verb!sze36.sh sze36_$mesh!
	\end{itemize}
\item run \verb!gnuplot sze36.plt!
\end{itemize}

\paragraph{Notes:}
\begin{itemize}
\item results with mesh \mesh{40}{40} and \mesh{48}{48} quite close to each other
\item results differ a bit from reference
\item the analysis does not converge about the snapthrough.
\end{itemize}
When the displacement of point A is imposed:
\begin{itemize}
\item the simulations run smoothly to the end (Figure~\ref{fig:sze36_d})
\end{itemize}


\subsection{Test 3.7: Pinched semi-cylindrical isotropic and laminated shells}

\paragraph{Data:}
\begin{itemize}
\item isotropic shell:
	\begin{itemize}
	\item $E$ = 2.0685$\times$10$^7$
	\item $\nu$ = 0.3
	\end{itemize}
\item laminated shell:
	\begin{itemize}
	\item $E_L$ = 2068.5 (note: probably $E_L$ = 2.0685$\times$10$^7$)
	\item $E_T$ = 517.125
	\item $G_{LT}$ = 795.6
	\item $\nu_{LT}$ = $\nu_{TT}$ = 0.3
	\end{itemize}
\item $R$ = 101.6
\item $L$ = 304.8
\item $h$ = 3
\item $P_\text{max}$ = 2,000
\end{itemize}
Note: some of the geometrical data is missing from \cite{SZE-2004};
it has been recovered from \cite{SZE-2002}.


\subsection{Test 3.8: Hinged cylindrical isotropic and laminated roofs}


\bibliographystyle{unsrt}
\bibliography{mybib}

\end{document}
